% this file is called up by thesis.tex
% content in this file will be fed into the main document

% Glossary entries are defined with the command \nomenclature{1}{2}
% 1 = Entry name, e.g. abbreviation; 2 = Explanation
% You can place all explanations in this separate file or declare them in the middle of the text. Either way they will be collected in the glossary.

% required to print nomenclature name to page header
\markboth{\MakeUppercase{\nomname}}{\MakeUppercase{\nomname}}


% ----------------------- contents from here ------------------------

% Math
\nomenclature{SVM}{Support Vector Machines, a set of learning methods used mainly for classification and regression.} 
\nomenclature{GPU}{Graphics Processing Unit, a special processing unit geared towards high-speed parallel processing. }
\nomenclature{GPGPU}{General-purpose computing on graphics processing units}
\nomenclature{SGD}{Stochastic Gradient Descent. Simple optimization method for minimizing an objective function that is written as a sum of differentiable functions.}
\nomenclature{VC Dimension}{Vapnik–Chervonenkis dimension is a measure of the capacity of a statistical classification algorithm, defined as the cardinality of the largest set of points that the algorithm can shatter.}
\nomenclature{RBF}{Radial basis function, one of the most popular kernel functions for SVM. }
\nomenclature{IP methods}{Interior point methods for convex optimization}
\nomenclature{SMO}{Sequential minimal optimization, a decomposition method for SVM training. }
\nomenclature{RKHS}{Reproducible Kernel Hilbert Spaces}
\nomenclature{ROI}{Region of Interest, an area of the image that is searched for object position}