% this file is called up by thesis.tex
% content in this file will be fed into the main document

\chapter{Conclusion} % top level followed by section, subsection

The main goal of the work presented in this thesis is development of the new online methods, and proving its applicability to the real-world tasks in image processing. After reviewing the state-of-the art methods, including online SVM training and boosting, we have developed our own method combining  effectiveness of the offline AdaBoost and convergence properties of the online methods like Pegasos. We have proposed a methods iteratively adding and training weak classifiers while at the same time updating the boosting coefficients combining them into a strong classifier with increased accuracy. The resulting algorithm can be used for both linear and non-linear classification, and exhibits high stability in regard to improperly tuned parameter, allowing to use the same parameter set for varying data distribution. Furthermore, it has lower computational and memory requirements than the the state-of-the art methods it was derived from. 
To show the effectiveness of the proposed algorithms have evaluated our method against Pegasos (\cite{Pegasos}), NORMA  (\cite{Norma}) and online boosting (\cite{OnlineBoost}), all methods quite popular in the scientific community for use with large datasets. We used several synthetic and publicly available natural datasets to show the stability and improved convergence rate of our method compared to other methods.

To further show the practical applicability of our algorithm, we have developed a simple tracking application for the mobile device that uses our learning method to learn and maintain object model. We have introduced a set of simple features sufficient to identify certain classes of object, and have shown that our method allows their combination into a relatively stable tracker that runs at near real-time speed (12-15 fps) even with unoptimized code. This application, then serves as a proof-of-concept application demonstrating the utility of our learning technique.


% ----------------------- paths to graphics ------------------------

% change according to folder and file names
\ifpdf
    \graphicspath{{7/figures/PNG/}{7/figures/PDF/}{7/figures/}}
\else
    \graphicspath{{7/figures/EPS/}{7/figures/}}
\fi


% ----------------------- contents from here ------------------------






% ---------------------------------------------------------------------------
% ----------------------- end of thesis sub-document ------------------------
% ---------------------------------------------------------------------------