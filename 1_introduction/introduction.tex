
% this file is called up by thesis.tex
% content in this file will be fed into the main document

%: ----------------------- introduction file header -----------------------
\chapter{Introduction}

% the code below specifies where the figures are stored
\ifpdf
    \graphicspath{{1_introduction/figures/PNG/}{1_introduction/figures/PDF/}{1_introduction/figures/}}
\else
    \graphicspath{{1_introduction/figures/EPS/}{1_introduction/figures/}}
\fi

% ----------------------------------------------------------------------
%: ----------------------- introduction content ----------------------- 
% ----------------------------------------------------------------------



%: ----------------------- HELP: latex document organisation
% the commands below help you to subdivide and organise your thesis
%    \chapter{}       = level 1, top level
%    \section{}       = level 2
%    \subsection{}    = level 3
%    \subsubsection{} = level 4
% note that everything after the percentage sign is hidden from output



\section{Aims of research} % section headings are printed smaller than chapter names
% intro
Recently, there has been several breakthroughs in the area of online learning and classification. This was partly driven by the massive amount and ever increasing speed of data acquisition in our information-driven society. The amount of data, especially in applications related to image processing quickly outstrips the capacity of many common learning algorithms that require all of the data to fit in memory or to be available at the same time. Futhermore, many of them have computational requirements that are polynomial of degree two or above on the amount of data. In particular, algorithms related to Support Vector Machines (SVM), a common and effective classification tool introduced by Vapnik {{=ref=}}, are usually quite computationally expensive.

For these reasons, several low-complexity, linear- or near-linear time algorithms has been developed specifically for the tasks that either have online limitations, i.e. only a limited number of samples available at the time and only sequential access to data, or need to process the amount of data that cannot fit in memory.

Two related types of classifier methods enjoy increased attention lately. For one, Support Vector Machines, primarily binary classifiers that have been successfully used for various tasks in image and signal processing, speech recognition and DNA analysis, have been successfully adapted to the online setting by using Stochastic Gradient Descent methods on their primal formulation. However, while incredibly efficient in the case of linear classification, introducing kernels to achieve nonlinearity in the online setting results in the  rapid increase of computational complexity, as kernel expansion coefficients are accumulated. 

On the othe hand, boosting algorithm for aggregating several simple classifiers into one stronger has also been adapted for various online and adaptation tasks, especially in the areas like object tracking in the video sequence, where the adaptability of a model to changing conditions is paramount. Most of such methods, however, limit the complexity, and, as a result, possible accuracy, by fixing the amount of added classifiers and fixing the feature pool. 

{\bf Our Goal} In this thesis we aim to bridge the gap between boosting and online SVM learning by exploiting the simolarities between both that allow us to introduce a new boosting algorithm based on two-step SVM training. The proposed algorithm allows for greater flexibility, and has smaller computational costs than traditional kernel-based methods while achieving similar or greater accuracy. To our knowledge this is the first such algorithm proposed. 

{\bf Contrbitions} Our contributions in this paper, are therefore as follows. First we propose a new learning method and compare it against existing methods in terms of accuracy and computational complexity, as well as proposing several variations of the method useable for various applications. Then, we introduce a specific application of the descibed method to the task of video tracking on the mobile device. We show that reduced computational costs of our method, as well as highly parallel processing on the device's GPU allows such complex applications to run in the real time. Also, for this application a new set of simple features is introduced and evaluated.

\section{Overview of common methods} % subsection headings are again smaller than section names
% lead
As mentioned above, the algorithm introduced in this paper is based partly on new SGD-based training methods, as well as well known algorithms, such as Adaboost. In particular, our work has been inspired by the following algorithms:

{\bf NORMA} Naive Online Risk Minimization Algorithm {{=ref=}} is one of the first and most generics algorithms based on stochastic gradient descent. It can be applied to various online task that require nonlinear separation of data, including kernel-based SVM, regression and novelty detecion. This paper has served as a basis to many other related algorithms, and provides a solid theoretical background by defining bounds on error and convergence rates of SGD-based methods.

{\bf Pegasos } Primal Estimated sub-GrAdient SOlver for SVM (PEGASOS) ({{==ref==}}) is a more recent algorithm that bridges the span between online and batch learning by allowing several samples to be processed at once. It gives significantly iproves convergence speed compared to NORMA at the price of a slight increase of a computational complexity of a single iteration. 

{\bf Online AdaBoost} In a series of papers, {{=ref=}}, and online boosting algorithm for feature selection is introduced, and several applications of it are discussed. This algorithm, and its limitation, are what has originall inspired us to work on a boosting-related methods.


\section{Algorithm outline}

Our algorithm takes at its basis the offline Adaboost algorithm and transfers it to online setting by utilizing its similarity to the SVM formulation. 

\section{Possible applications}
In the introduction, we briefly state the need and motivation of our research. We describe the current state of the art techniques for the SVM training and boosting, and show some of their drawback that we aim to address by introducing our method. We then briefly outline our algorithm and its relation to the previously described algorithms. We further describe possible applications of our algorithm, particularly in the field of image processing.



% dextran, starch, glycogen


%: ----------------------- HELP: special characters
% above you can see how special characters are coded; e.g. $\alpha$
% below are the most frequently used codes:
%$\alpha$  $\beta$  $\gamma$  $\delta$

%$^{chars to be superscripted}$  OR $^x$ (for a single character)
%$_{chars to be suberscripted}$  OR $_x$

%>  $>$  greater,  <  $<$  less
%≥  $\ge$  greater than or equal, ≤  $\ge$  lesser than or equal
%~  $\sim$  similar to

%$^{\circ}$C   ° as in degree C
%±  \pm     plus/minus sign

%$\AA$     produces  Å (Angstrom)




% dextran, starch, glycogen continued

%: ----------------------- HELP: references
% References can be links to figures, tables, sections, or references.
% For figures, tables, and text you define the target of the link with \label{XYZ}. Then you call cross-link with the command \ref{XYZ}, as above
% Citations are bound in a very similar way with \cite{XYZ}. You store your references in a BibTex file with a programme like BibDesk.


%: ----------------------- HELP: adding figures with macros
% This template provides a very convenient way to add figures with minimal code.
% \figuremacro{1}{2}{3}{4} calls up a series of commands formating your image.
% 1 = name of the file without extension; PNG, JPEG is ok; GIF doesn't work
% 2 = title of the figure AND the name of the label for cross-linking
% 3 = caption text for the figure

%: ----------------------- HELP: www links
% You can also see above how, www links are placed
% \href{http://www.something.net}{link text}


% variation of the above macro with a width setting
% \figuremacroW{1}{2}{3}{4}
% 1-3 as above
% 4 = size relative to text width which is 1; use this to reduce figures

%: ----------------------- HELP: lists
% This is how you generate lists in LaTeX.
% If you replace {itemize} by {enumerate} you get a numbered list.


 


%: ----------------------- HELP: tables
% Directly coding tables in latex is tiresome. See below.
% I would recommend using a converter macro that allows you to make the table in Excel and convert them into latex code which you can then paste into your doc.
% This is the link: http://www.softpedia.com/get/Office-tools/Other-Office-Tools/Excel2Latex.shtml
% It's a Excel template file containing a macro for the conversion.



% There you go. You already know the most important things.


% ----------------------------------------------------------------------



